% document based on the VU Beta / BSc Thesis template

\documentclass[11pt]{article}
\usepackage{graphicx}
\usepackage{hyperref}

      \textwidth 15cm
      \textheight 22cm
      \parindent 10pt
      \oddsidemargin 0.85cm
      \evensidemargin 0.37cm

%
\usepackage{xcolor}  % Colored text etc.
%
\definecolor{OwnAzure}{HTML}{336699}
\definecolor{OwnCerulean}{HTML}{CAE2FE}
\definecolor{OwnOliveGreen}{HTML}{556B2F}
%
\usepackage[colorinlistoftodos,prependcaption,textsize=tiny]{todonotes}% Add ,disable to the options, to hide the comments
%
%
\usepackage{xargs}                      % Use more than one optional parameter in a new commands
%
\newcommand{\todojacob}[1]{\todo[inline, color=OwnAzure!40]{Jacob: #1}}
\newcommand{\todoalexandru}[1]{\todo[inline, color=orange!40]{Alexandru: #1}}

\newcommand{\thoughts}[1]{\todo[inline, color=OwnOliveGreen!40]{Question: #1}}
\begin{document}

\thispagestyle{empty}

\begin{center}

Vrije Universiteit Amsterdam

\vspace{1mm}

\includegraphics[height=28mm]{vu-griffioen-white.pdf}

\vspace{1.5cm}

{\Large Bachelor Thesis}

\vspace*{1.5cm}

\rule{.9\linewidth}{.6pt}\\[0.4cm]
{\huge \bfseries Title of the Research Project\par}
{\huge \bfseries Comes Here\par}\vspace{0.4cm}
\rule{.9\linewidth}{.6pt}\\[1.5cm]

\vspace*{2mm}

{\Large
\begin{tabular}{l}
{\bf Author:} ~~Jacob Burley ~~~~ (2599965)
\end{tabular}
}

\vspace*{1.5cm}

\begin{tabular}{ll}
{\it 1st supervisor:}   & ~~dr. Alexandru Iosup \\
{\it 2nd reader:}       & ~~Laurens Versluis
\end{tabular}

\vspace*{2cm}

\textit{A thesis submitted in fulfillment of the requirements for\\ the VU Bachelor of Science degree in Computer Science}

\vspace*{1cm}

\today\\[4cm] % Date

\end{center}

\newpage

%Notes:
%\begin{enumerate}

%    \item Curious about guidelines on conducting and on supervising research projects for young researchers? %For both students and supervisors, we recommend consulting guidelines on conducting and on supervising research projects for young researchers, respectively. 
%    There are several options to learn more, for example, from this book by Sharp et al.~\cite{research:book/SharpPW02}.

%    \item New to \LaTeX{}? Consult a quick guide, such as~\cite{techrep:latex,techblog:latex}.
    
%    \item Is this going to be very abstract? We use as a running example a project resulting in a Tier-1 publication~\cite{DBLP:conf/sc/AndreadisVMI18}. 
    %The project took the HP student about 25~ECs to complete, including the conference presentation and the polished, camera-ready version of the article. (Take these numbers with a grain of salt---without detracting from their achievement, the student received plenty of support from both the supervisor and the research team.)
    
%\end{enumerate}


%\section*{Important}

%We do not have a lower-limit on the number of pages, but we do have one for the upper-limit. {\bf Keep the problem statement to under 4 pages, not including the title-page, the plan, and the references.}



\section*{Abstract}
Will be written last.

\section{Introduction} \label{sec:introduction}
	With the rapid expansion in the use of cloud services by both consumers and businesses \cite{Kushida2015}\cite{mokhtar2013}, the data centers that house these services must also grow. 
	While many may not see the cloud in its physical manifestation, cloud services are provided by servers, and servers are typically stored in data centers. 
	The configuration of such servers varies drastically, depending on the intended workload. 
	Relational databases, for example, can have large memory requirements. 
	Likewise, file servers may not need a particularly powerful CPU, but do require a lot of disks, and perhaps even a network card with higher throughput.

	When architecting data centers, many stakeholders must come together and agree on the various specifications of the data center, each with their own unique set of requirements. 
	System architects will then begin to design a system that meets these requirements. 
	However, such projects also have constraints. 
	Designing a data center isn't as easy as buying servers that meet the requirements of the stakeholders and putting them into use. 
	Constraints such as budget or physical space often influence design decisions, with smaller data centers perhaps comprising of higher-density compute nodes. 
	Designing data centers is often a complex problem, for which no validated analytical models exist. 
	Despite some academic institutions often being forthcoming with the technologies used in their data centers, there is no one-size-fits-all approach for data center design, making it hard to determine what hardware you need for a given workload.

	We focus in this work on OpenDC, an open-source data center simulator \cite{Iosup2017}, designed to simulate a wide range of workloads across a variety of server hardware. 
	In its current state, OpenDC can be used to simulate user-specified workloads on user-designed systems. 
	Users must specify each individual component used, and build out an entire system. 
	This requires a high level of technical understanding, and relies solely on the user's knowledge of the intended workload when component choices are made. 
	In this way, the reasoning behind component choices has to be expressed by the individual, rather than OpenDC itself. 

	Looking forward, it would be beneficial for OpenDC to simulate user-specified workloads on systems designed by OpenDC itself. 
	OpenDC would be able to leverage a large database of performance data to make component choices based on the workload specified by the user. 
	In this way, the system would be designed for the best objective performance at the specified workload, potentially adherent to specified constraints such as financial or power budget. 

	With the implementation of prefabs, this process would also become transparent to the user. 
	A user could ask OpenDC to design a complete system for a specified workload, but they could also design their own system using component recommendations from OpenDC. 
	These component recommendations would come in the form of prefabs, collections of components that together are performant at a specific task. 
	A pre-fab may be as small as one server, but could also be a rack full of servers, or even a room full of equipment. 

	This process would be akin to the modularity already seen in software engineering, where a programmer may import a library (often written by someone else) to enable certain functionality in their program. 
	The programmer often only uses one or two functions from this library, and do not necessarily understand how these functions work (nor do they need to).

	This concept is already offered by some cloud providers: DigitalOcean is a Virtual Private Server (VPS) provider that harnesses a "marketplace" of pre-configured VPS templates (known as "1-Click Apps") in order to simplify use of the platform \cite{DigitalOcean2020}. 
	Customers can then easily add a pre-configured VPS to their environment, without having to worry about operating system installation, configuration, or maintenance. 
	Templates provided through the platform are typically created by the vendors of the software used in each template, and are thus of high quality and suitability for the application.

\newpage

\section{Background} \label{sec:background}

	\subsection{The openDC project}
		OpenDC is an open-source Discrete Event Simulator used to simulate performance characteristics of workloads in massive computer systems. 
		Its purpose is to assist in performance testing of data centers designed by users, with the intention that the information gained can be factored in when these systems are being designed. 
		OpenDC simulates workloads using traces from the Grid Workload Archive \cite{Iosup2008}, which are provided by participants who contribute to the archive. 
		These traces provide metadata about the scheduling requirements of jobs frequently sent to data centers, such as CPU usage throughout a job. 
		Using this information, OpenDC can simulate the speed with which a workload will complete, as well as resource usage throughout the workload. 
		These simulations can assist businesses and educational institutions in highlighting and addressing performance concerns in their next-generation environments before any hardware has even been purchased.
	\subsection{Modularity in other areas of computing}
		In software architecture, it is becoming increasingly common to rely on frameworks and libraries written by others in the software engineering community. 
		It is usually not necessary to understand the full workings of such a library, as the benefit comes from its utility in meeting certain requirements. 
		We can view these software modules as prefabricated software. 
		It follows, then, that it would be helpful to be able to describe datacenter hardware in such a way. 
		With the advent of PaaS and particularly SaaS offerings from major cloud providers, developers can implement entire layers of their software stack with just a few clicks. 
		These layers still run in virtual machines, but the developer does not need to concern themselves too much with building or maintain the hardware or operating system.
	\subsection{The goal of prefabs}
		Users of OpenDC would benefit from prefabs in much the same way, only having to focus on which services they want to run in their datacenter.
		Large datacenter providers using OpenDC may also see a benefit. 
		Cloud service providers who provide IaaS services typically use homogenous hardware configurations, with different configurations for each performance tier. 
		With prefabs, it would be straightforward to create a prefab that is representative of a given performance tier, and then clone it when performing capacity planning during periods of growth.


\section{A Design for a Representation of Data Center Hardware}

	\subsection{A model for datacenter representation}
		In this research, we provide a model for representing datacenter hardware. 
		This model is not the first of its kind: Andreadis et al (2018) created a model for datacenter hardware in order to model scheduling in OpenDC. 
		Our model, however, offers more detail. 
		When designing the model, it is important to consider that a large part of its purpose is to increase usability of OpenDC. 
		As a result, we choose to represent lots of hardware characteristics that are not used by the simulator, but provide useful information to the user. 
		Brands of hardware are not necessary in order to simulate workloads, but they are useful when a user is making design decisions, or presenting their design to a wider audience.
	\subsection{A definition and corresponding design of a datastructure for storing prefabs}
		In order to store our datacenter hardware representation within OpenDC, we define a datastructure to represent the datacenter hardware. 
		This datastructure aims to be simple to understand, as well as easily expandable in order to represent hardware configurations that we have not focussed on in this research (i.e. blade servers, or other chassis that may contain multiple/unconventional motherboards). 
		For this reason, we have chosen to use JavaScript Object Notation (JSON) to store our datastructure. 
		This supports our goals of ease of use and expandibility, as JSON is both human-readable, 
	\subsection{Designing ways to create and interact with prefabs (in OpenDC)}

\section{Implementation of a Prototype}

	\subsection{Why prototype?}
		Prototyping is an important part of our design process for implementing the new topologies in OpenDC.
		The design proposal requires the implementation of certain new technologies (such as MongoDB) which we are unfamiliar with.
		As a result, prototyping provides us with a way of becoming familiar with these new technologies, as well as assessing their suitability, before we begin the process of implementing them into the existing OpenDC codebase.
		We also can create prototypes of the datastructure, as well as the corresponding interactions with it, and assess the suitability of the datastructure with regards to the interactions we require it to be capable of handling.
	\subsection{Creating a prototype}
		The two main learning outcomes of prototyping were to explore how we could best implement and interact with MongoDB, and to iteratively improve our datastructure designs.
		As a result, the prototype consisted of three components: a MongoDB instance, a Python module that interfaced with the database, and a second Python application to serve as a rudimentary frontend. 
		This configuration was chosen to be relatively close to how OpenDC already implemented its database connections, which was structured in a similar way. 
	\subsection{The process of implementing the prototype}

	\subsection{Prototype validation}

	\subsection{Lessons learned}
		During prototyping, we learnt many things that influenced the decisions we made during the implementation phase.
		We also learnt a lot about the differences between MongoDB and SQL, which allowed us to improve how we implement this new database technology throughout OpenDC.

\section{Evaluation of Suitability of Design \& Implementation}


\section{Conclusion} \label{sec:conclusion}

% For more on bibliography styles, see 
% https://www.overleaf.com/learn/latex/Bibtex_bibliography_styles
\bibliographystyle{abbrv}
\bibliography{main}


\end{document}
% \end{document}



